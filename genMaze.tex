\documentclass{article}

\title{Maze Generation by Random Walks}
\author{Adrian-Horia Dediu}
\begin{document}

\maketitle
\section*{Introduction}
This document outlines the procedure to generate mazes using random walks. It is intended to document the code from generate\_maze.py


\section*{Formal Definition of a Maze}
We briefly introduce the notions of graphs and mazes following the notations from ...

A \emph{graph} $g$ is a tuple $g=(V,E)$ where:
\begin{itemize}
  \item[] $V$ is a set of vertices,
  \item[] $E=\{\{v_1,v_2\} \,|\, v_1, v_2 \in V\}$ is a set of edges.
\end{itemize}

A \emph{maze} $m$ is a tuple $m=(V,E,b,e)$ where:
\begin{itemize}
  \item[] $(V,E)$ is a graph,
  \item[] $b,e$ are the start and end vertices of the maze respectivelly.
\end{itemize}

We focus on \emph{connected mazes}, where there exists a path between any pair of vertices. For all $v_1, v_2 \in V$, there exists a path in the graph from $v_1$ to $v_2$.


\section*{The Random Walk Algorithm}

A random walk is a mathematical formalism that describes a succession of random steps within a mathematical space. Here, we will leverage this principle to create an algorithm that generates mazes.

The steps involved in this process are outlined below:

\begin{enumerate}
    \item \textbf{Initialization:} Select an arbitrary, open cell in the maze surface to start from and designate it as visited. This single node forms our initial tree \( T \), i.e., \( T = (V', E') \) where \( V' \) consists of the initial cell and \( E' \) is initially empty.

    \item \textbf{Walk construction:} From the current cell, walk randomly to an open neighbouring cell. If the neighbouring cell has not been visited before, add it to \( V' \) and the line joining the neighbouring cell and the current cell is added to \( E' \).

    \item \textbf{Continuation:} Continually perform the step above until a previously visited cell is encountered.

    \item \textbf{Unvisited Detection:} In the case that the walk construction comes to a halt, select an unvisited cell in the maze and begin another random walk as described in step 2.

    \item \textbf{Walk Addition:} As the new walk is developed, append the nodes and the edges connecting them to the respective \( V' \) and \( E' \).

    \item \textbf{Interative Process:} Repeat Step 4 and 5 until all cells in the maze have been visited. This step assures that a complete and unbroken graph labyrinth has been created.
\end{enumerate}

This iterative process guarantees that a path exists between every pair of cells in the labyrinth.

\end{document}